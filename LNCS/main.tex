\documentclass[envcountsect]{llncs}
\bibliographystyle{splncs04}
\pagestyle{headings}

% = = = = = Packages = = = = = %

% !TEX root = ../main.tex

%------------------------LNCS----------------------%

%None

%------------------------Packages----------------------%

% = = = Graphics
\usepackage[pdftex]{graphicx}
\graphicspath{{figures/}}
\DeclareGraphicsExtensions{.jpg,.png}
\usepackage[table,xcdraw]{xcolor}

% = = = Subfig (note not subfigure)
\usepackage[caption=false,font=footnotesize]{subfig}

% = = = Math Symbols
\usepackage{amsmath}
%\usepackage{amstext,amssymb,amsthm}
\usepackage{bbm}
\usepackage{stmaryrd}
\usepackage{wasysym}
\usepackage{amssymb}
\usepackage{color}
\usepackage{multirow}
\usepackage{rotating}
\usepackage{makecell}
\usepackage{hhline}
\usepackage[misc]{ifsym}
% = = = Other
\usepackage{array}
\usepackage{color}
\usepackage[hyphens]{url}
\usepackage[pdftitle=Title,pdfauthor=Anonymous]{hyperref}
\usepackage{comment}
%------------------------END----------------------%  
  


\input{setup/macros}
% !TEX root = ../main.tex

\begin{table}[t!]
    \centering
    
        \begin{tabular}{lllllllllllllll}
    
    &
    \headrow{Can replace entire logic} &
    
    \headrow{can replace pre-specified part of logic} & 


    \headrow{Can replace entire state} &
    
    \headrow{can change pre-specified state variables} &
    
    \headrow{No need to deploy a new contract} &

    \headrow{No need to migrate state from old contract} &

    \headrow{No need to separate State and Logic} &

    \headrow{Function Selector Clashes Risk} &

    \headrow{Storage Clashes Risk} & 

    \headrow{No indirection} & 


    \headrow{User endpoint address not changed} &
    

    \headrow{Downtime in upgrade events} &

    \headrow{No need to change code to add the upgrade pattern} &

    \headrow{Need to change a state variable} 
    
    
    \\
    
    \hline
 
    
        \multicolumn{1}{c|}{Parameter Configuration}	& \multicolumn{1}{c|}{}  & \multicolumn{1}{c|}{} &  \multicolumn{1}{c|}{} & \multicolumn{1}{c|}{\checkmark} & \multicolumn{1}{c|}{\checkmark} & \multicolumn{1}{c|}{\checkmark} &  \multicolumn{1}{c|}{\checkmark} &  \multicolumn{1}{c|}{} & \multicolumn{1}{c|}{} & \multicolumn{1}{c|}{\checkmark} & \multicolumn{1}{c|}{\checkmark} & \multicolumn{1}{c|}{} &\multicolumn{1}{c|}{\checkmark} & \multicolumn{1}{c}{\checkmark}\\
    
        \hline
  
        \multicolumn{1}{c|}{Component Change}	& \multicolumn{1}{c|}{}  & \multicolumn{1}{c|}{\checkmark} &  \multicolumn{1}{c|}{} & \multicolumn{1}{c|}{} & \multicolumn{1}{c|}{} & \multicolumn{1}{c|}{\checkmark} &  \multicolumn{1}{c|}{\checkmark} &  \multicolumn{1}{c|}{} &  \multicolumn{1}{c|}{} &  \multicolumn{1}{c|}{\XBox} & \multicolumn{1}{c|}{\checkmark} & \multicolumn{1}{c|}{} & \multicolumn{1}{c|}{\XBox} &  \multicolumn{1}{c}{\checkmark}\\
        

        \hline

        \makecell{Contract Migration}	& \multicolumn{1}{|c|}{\checkmark}  & \multicolumn{1}{c|}{} &  \multicolumn{1}{c|}{\checkmark} & \multicolumn{1}{c|}{} & \multicolumn{1}{c|}{} & \multicolumn{1}{c|}{} &  \multicolumn{1}{c|}{\checkmark} &  \multicolumn{1}{c|}{} &  \multicolumn{1}{c|}{} &  \multicolumn{1}{c|}{\checkmark} & \multicolumn{1}{c|}{} & \multicolumn{1}{c|}{} & \multicolumn{1}{c|}{\checkmark} &  \multicolumn{1}{c}{}\\
    
         \hline
         \makecell{Create2 metamorphosis}	& \multicolumn{1}{|c|}{\checkmark}  & \multicolumn{1}{c|}{} &  \multicolumn{1}{c|}{\checkmark} & \multicolumn{1}{c|}{} & \multicolumn{1}{c|}{} & \multicolumn{1}{c|}{} &  \multicolumn{1}{c|}{\checkmark} &  \multicolumn{1}{c|}{} & \multicolumn{1}{c|}{}&  \multicolumn{1}{c|}{\checkmark} & \multicolumn{1}{c|}{\checkmark} & \multicolumn{1}{c|}{\checkmark} & \multicolumn{1}{c|}{\checkmark} &  \multicolumn{1}{c}{}\\

    
        \hline
        \makecell{Consensus Override}	& \multicolumn{1}{|c|}{\checkmark}  & \multicolumn{1}{c|}{} &  \multicolumn{1}{c|}{\checkmark} & \multicolumn{1}{c|}{} & \multicolumn{1}{c|}{} & \multicolumn{1}{c|}{} &  \multicolumn{1}{c|}{\checkmark} &  \multicolumn{1}{c|}{} & \multicolumn{1}{c|}{}&  \multicolumn{1}{c|}{\checkmark} & \multicolumn{1}{c|}{\checkmark} & \multicolumn{1}{c|}{\checkmark} & \multicolumn{1}{c|}{\checkmark} &  \multicolumn{1}{c}{}\\

        \hline

        \makecell{Call-based}	& \multicolumn{1}{|c|}{\checkmark}  & \multicolumn{1}{c|}{} &  \multicolumn{1}{c|}{} & \multicolumn{1}{c|}{} & \multicolumn{1}{c|}{} & \multicolumn{1}{c|}{\checkmark} &  \multicolumn{1}{c|}{} &  \multicolumn{1}{c|}{} &  \multicolumn{1}{c|}{} &  \multicolumn{1}{c|}{} & \multicolumn{1}{c|}{} & \multicolumn{1}{c|}{} & \multicolumn{1}{c|}{} & \multicolumn{1}{c}{\checkmark}\\        
        
         \hline

         \makecell{DelegateCall-based}	& \multicolumn{1}{|c|}{\checkmark}  & \multicolumn{1}{c|}{} &  \multicolumn{1}{c|}{} & \multicolumn{1}{c|}{} & \multicolumn{1}{c|}{} & \multicolumn{1}{c|}{\checkmark} &  \multicolumn{1}{c|}{} &  \multicolumn{1}{c|}{\checkmark} & \multicolumn{1}{c|}{\checkmark}&  \multicolumn{1}{c|}{} & \multicolumn{1}{c|}{\checkmark} & \multicolumn{1}{c|}{} & \multicolumn{1}{c|}{\XBox} & \multicolumn{1}{c}{\checkmark}\\    
        
         \hline
        
        
        \end{tabular}
        \captionsetup[tabular]{singlelinecheck=off}
        \caption{Evaluation\ref{tab:eval}}
       
    
    \end{table}
    \footnotetext[1]{Design of system in which a parameter can change the logic is hard}

%        \makecell{Diamonds}	& \multicolumn{1}{|c|}{\checkmark}  & \multicolumn{1}{c|}{} &  \multicolumn{1}{c|}{} & \multicolumn{1}{c|}{} & \multicolumn{1}{c|}{} & \multicolumn{1}{c|}{\checkmark} &  \multicolumn{1}{c|}{} &  \multicolumn{1}{c|}{\checkmark} & \multicolumn{1}{c|}{\checkmark\checkmark}&  \multicolumn{1}{c|}{} & \multicolumn{1}{c|}{\checkmark} & \multicolumn{1}{c|}{} & \multicolumn{1}{c|}{\XBox} & \multicolumn{1}{c}{\checkmark}\\


%\usepackage[all=normal,floats=tight,paragraphs=tight]{savetrees}
%\usepackage{fullpage}

% = = = = = Title = = = = = %

\begin{document}
\frontmatter
\mainmatter

\title{Not so immutable: Upgradeability of Smart Contacts on Ethereum}
\author{Mehdi Salehi\inst{1} \and Jeremy Clark\inst{1} \and Mohammad Mannan\inst{1}}
\institute{Concordia University, Montreal, Canada}

\maketitle


% = = = = = Abstract = = = = = %

% !TEX root = ../main.tex

\begin{abstract}

In this paper we explore how developers torture an immutable blockchain into allowing smart contracts to be upgraded. 

\end{abstract}

% = = = = = Main Body = = = = = %

% !TEX root = ../main.tex

% = = = = = = = = = = = = = = = = = = = = = = = = = = = = = = = = = = = = = = = = = =

\section{Introductory Remarks}



\section{Classification}
% We have some off-chain upgrades like what people did to force UNISwap using arbitrum as the L2 solution. It is an off-chain upgrade not an automated one.
% Does Factory patterns can be defined as upgradeability patterns? Like creating a new pool for Uniswap!
\subsection{Retail Changes}
This is not a standardized pattern. The development team must consider the ways to upgrade the contracts before deploying the smart contract. Known patterns are different on the level of intervention to change the logic that they need to change in the future. The amount of changes are limited and system design can not be changed after deployment and just some system variables can be changed. We will describe three famous patterns here:

\paragraph{Parameter Configuration}. The easiest way to upgrade the logic of the smart contract is to have some critical parameter that can change the whole logic of the system. For instance, in economy we have different variables which have effect in the interest rates. By changing those variables the governors will response to changes needed for the system. 
In this model we have a setter function to change the upgradable parameters if the system needs upgrades. The best example for this type of upgradeability is MakeDao project. In Maker there are some variables like Dai Saving Rate (DSR) or Stability fee that can be changed through governance vote. The logic behind the smart contract and the tokenomic of the Dapp completely depends on these variables.

\paragraph{Strategy pattern}. The strategy pattern is an easy way for changing part of the code in a contract responsible for a specific feature. Instead of implementing a function in your contract to take care of a specific task, you call into a separate contract to take care of that – and by switching implementations of that contract, you can switch between different strategies.
An example for this pattern is Compound project and how they used strategy pattern for their interest rate model. There is a interest rate model contract in Compound that can be changed during the time. 

\paragraph{Pluggable Modules}. In this pattern we have a core contract that have some immutable features and then new contracts generated by the main contract and each have some or all features of the main contract. This pattern is mostly used in wallets and DeFi services like DeFi saver and InstaDapp. Users can decide to add new features into their wallet. 

%TODO: Ask shayan about DeFi save and how do they upgrade their contracts.
\subsection{Wholesale Changes}
In contrast to previous session, sometime we need to change the whole or a big part of the logic of our smart contract. This update could be a response to an incident happen to the smart contract or a planed upgrade of the system. Before deployment, the core developer team should have a plan for the upgrade events. We categorized these types into two main classes:

\subsubsection{Contract Migration}
%who will push the old data? user to provide its own data to the new version?
In the migration plan we should write a completely new contract with our desired new logic. In migration method our new version contract doesn't have any communication with the previous versions. The challenges we face in migration method are:
\begin{enumerate}
    \item Grab the needed data (from previous contract or new data): 
    It depends on the data type. It is easy for simple data structures (\eg uint, address, or even arrays) to collect the data just by reading storage slots from the 0 slot. we should take care of complex structures (\eg mapping) in the latest versions of our contract by adding event updates whenever a data added to a mapping variable. In case of an upgrade we can use Logs to find storage slot (using key hash) and collect the data.
    Sometime we need to push new data into our upgraded smart contract. For example in airdrops we need new coin to be added to some specific addresses. 
    \item Push the data into the new contract: 
    Using Constructor, we can use batch transfer function with arrays of addresses and amounts as inputs. This way we can push lots of data using a single transaction. One limitation here is block gas limit. If we exceeds the block gas limit we need to push all data in different blocks (pausing in the first block and unpause at the end). Recently, Devs are using merkle distribution tree to push data on to the smart contracts.
    The most important thing here is the cost of pushing data to the new version. It depends on 1)the number of storage slots to be updated and 2)Method used to push the data on-chain. (Can be tested). 
    \item Stop the previous contract:
    Suppose that we have a token contract and we want to migrate to a new version. We should be confident that nobody can use the previous contract. If not, it is possible that a person sell a token from previous contract (which should be valueless after migration) to a person who doesn't aware of the migration plan.Because of the decentralized nature of the blockchain and Dapps you cannot reach to your customers to alert them from using the previous contract. One way to do that is have a pause option the your contracts and pause the old versions before migration.
  \end{enumerate}

Contract Migration is less riskier than other types of upgrades, not cost effective compare to some upgradeability types but more decentralized to the other solutions. Also, it's not good for frequent updates.The other advantage of this method is that it removes transaction gas cost needed for patterns like proxy, registry or call-based methods.

%https://blog.trailofbits.com/2018/10/29/how-contract-migration-works/
% Migration 300,000 balances = $7500 in october 2018



\subsubsection{Data Separation patterns}
The other type of wholesale methods is to separate data and logic part of our codes. In the case of the upgrade we can  keep the storage contract and just upgrade the logic contract and link the new version of logic contract to the storage contract.
There is a debate on whether this type of upgradeability is cheaper or not in comparison to migration method. But, this method is more efficient for Dapps in which we need frequent updates. The other important issue here is who decides on the changes we need for the system which we will discuss on further sections.
Here we have 2 different choices using Call method and Delegate Call method to link storage and logic contracts together.

\paragraph{Call based patterns}

In this type the interaction between logic and storage contact is handle by Call opcode in Ethereum Virtual Machine. In call based patterns user is supposed to call the logic contract and the logic contract will call the storage contract. The logic contract is the one that can be upgraded.

There are two concerns in this approach: how to store data and how to perform an upgrade.

\textit{Storage. } The easiest way to store data in storage contract is to have a modifier on the setter functions in the storage contract that allow just the logic contract to change the variables. The owner of the contract can change the address of logic contract for the modifier.
In this approach for adding a new persistent variable, a new data contract should be deployed which may be costly in case that the application needs lots of upgrades.

The other way to store data is so called Eternal storage (ERC930). Eternal storage uses mapping (key-value pair) to store data, using one mapping per type of variable. The EVM storage layout and how it handles mapping helps the Eternal storage pattern to be more amenable to evolution but also more complex.

\textit{Upgrade implementation. } There are three main ways to implement upgrades using data separation pattern. The easiest way is to change the ownership of storage contract into new upgraded logic contract and then \textbf{Pause} the old contract or set its pointer to 0x0 address. The other solution is to forward the calls receive by the old contract into the new logic contract. The last option is to set a proxy contract that just keeps the address of logic contract and call into logic contract.

% \textit{Risks. } (From trail of bits blog)
% "We have repeatedly seen clients deploy this pattern incorrectly. For example, one client’s implementation achieved the opposite effect, where a feature was impossible to upgrade because some its logic was located in the data contract.

% In our experience, developers also find the EternalStorage pattern challenging to apply consistently. We have seen developers storing their values as bytes32, then applying type conversion to retrieve the original values. This increased the complexity of the data model, and the likelihood of subtle flaws. Developers unfamiliar with complex data structures will make mistakes with this pattern."


\paragraph{DelegateCall-based upgrades}
Similar to call based patterns here we have two contracts, Storage and Logic contract. we may have more than one logic contract. The difference here is that the user is calling storage contract first(called proxy contract), and the proxy contract DelegateCalls to the logic contract(s).
The main difference between delegatecall and call-based approach is that in delegatecall proxy pattern the storage layout of proxy and logic contract should be the same. The difference between storage layouts will result in storage clashes. 
There are three different methods to mitigate the risk of storage clashes:

\textit{Inherited Storage}. 
In this method the proxy contract and all logic contracts are inherited from a storage contract that contains storage variables. Using this method we are confident that the proxy and logic contracts are using the same storage layout and storage clashes will be mitigated.
After deployment if we need new logic contract with new storage variables, we should deploy a new storage contract that inherits the previous storage contract. Then the new logic contract must inherit the new version of the storage contract.

This method is not efficient because of variables that declared but not used in some logic contracts. On the other hand, each logic contract is coupled with a storage contract and it is hard to take care of this track. 

\textit{Eternal Storage}. 
In this pattern, we defined mappings for all variable types that we need to use in our logic smart contract. For storing mapping variables EVM selects random slots on the storage based on the variable's name so we can mitigate the clashes using this randomness.

The main problem of this type is that the logic contract and all other contracts that are using the storage must use the mapping structure to access the storage variables and use complex syntax whenever they want to access a variable. This also results in the gas usage inefficiency because we need to call and update a mapping each time we need to change a variable.  

Also it is hard to use eternal storage for complex variables like mappings and structure (need mapping of mapping pattern). Also finding a state variable of the proxy smart contract is hard because we store them in arbitrary slots of the storage.
%  we cannot add new state variables using this method.

\textit{Unstructured Storage}. 
The other way of mitigating the storage clashes is to assign some randomly selected slots to critical variables like address of logic contract. For instance, open zeppling uses hash of "org.zeppelinos.proxy.implementation" to store the address of the logic contract in this slot.

The downside of this approach is that we need getter and setter function for each variable. We also can use unstructured storage for simple variables and not for mapping and structures.



\begin{figure}
  \centering
    
      \includegraphics[width=0.9\textwidth]{figures/Chart.png}
  \caption{Classification}
 \end{figure}

 \section{Evaluation of different methods}
 
 In this section we compare and evaluate different methods discussed in previous section. There are some characteristics that can help the designer to decide which method should be used on the system and add upgradeability to the Dapp.

 For our evaluation framework, we provide the definition of each evaluation criteria (i.e., column of the table), specifying what it means to receive a check mark (\checkmark) or nothing.

 %separate rows
% \subsection{Extent of Upgradeability}
 
 
 
\subsubsection{Can replace entire logic}

An upgradeability method in which the upgrader is able to replace the entire logic of the system earns a check mark(\checkmark) otherwise it receives nothing.

\subsubsection{can replace pre-specified part of logic}

An upgradeability method in which the upgrader can change \emph{just} pre-specified part of logic of the system (and not entire logic) earns a check mark(\checkmark) otherwise it receives nothing.

\subsubsection{Can replace entire state}

An upgradeability method that gives the upgrader ability to replace or transfer the entire state to the newer version earns a check mark(\checkmark) otherwise it receives nothing.

\subsubsection{can change pre-specified state variables}
 
An upgradeability method in which the upgrader can just change some pre-specified state variables receives a check mark (\checkmark) otherwise it awarded nothing.
 

 % JC: Be more precise about "whole logic" and "part". This can be broken into sub-columns if necessary. Start by defining the parts of a DApp: logic, state, etc. Then properties can be "Can replace logic" "Can replace state" "Can set state variables" (I don't know for sure, this is just ideas). This column is super important so make it as precise as you can. 
 


%  \textit{Retail Change} methods limit the the upgrades to a small part of the system (\Circle). However, using \textit{Migration}, \textit{Call-based} and \textit{DelegateCall-based} approaches, the upgrader is able to deploy a new smart contract with a new set of functions and logics.
%Add rows:
% Parameter change: can State variable change
% Strategy pattern: can part of Logic change


%%%%%%%%%%%%
% \subsection{Upgrade Frequency} \label{upgradeFRQ}
%%%%%%%%%5

\subsection{No need to deploy a new contract}

Using some upgradeability patterns, the upgrader needs to deploy a new smart contract in the process of upgrade which receives nothing. Upgradeability methods which do not need to deploy a new contract receive a check mark (\checkmark).

\subsection{No need to migrate state from old contract}

In some patterns, the upgrader needs to collect and transfer old data from previous version to the newer one which receive nothing (\checkmark) otherwise it receives a check mark (\checkmark). 
% JC: I would start the clock at "decision to upgrade" rather than proposal (otherwise is depends on how decisions are made). Also is single transaction the right metric? I can write an upgrade helper contract to do all the steps in a single function call. I think it could be better as: do you have to push a NEW contract? Do you have to change (more than one?) EXISTING contracts?

% An upgradeability method that can move from proposal to finality within a single transaction is awarded a full dot (\CIRCLE). A process that requires more than a single block time in order to write and deploy a new logic contract receives half circle (\LEFTcircle). A method that needs more than just deploying a new logic contract and changing some parameters awarded empty circle (\Circle).

% The upgrader can process the whole upgrade using \textit{Retail changes} method in a single transaction time window after proposing the change (\CIRCLE). On the other hand, \textit{Call-based} and \textit{DelegateCall-based} approaches are more time consuming because the upgrader needs a time window to write the new logic contract, deploy it to the mainnet, and change some address pointers (\LEFTcircle). \textit{Migration} method is more time consuming and not good for frequent changes because Before the upgrade the developers need to collect data of old contract (\eg getting storage snapshot), write a new version of the logic smart contract, and push the old data into the newer version which is time consuming and sometimes it takes multiple blocks to push all data from the old version (because of block gas limit of a block) (\Circle).
 %%%%%%%%%%
%  \subsection{System Complexity}
 %%%%%%%%%%%%%
\subsection{No need to separate State and Logic}

An upgradeability pattern that does not requires separation of logic and storage contracts awarded a check mark (\checkmark) otherwise it receives nothing.

\subsection{Not using DelegateCall opcode}

Some of the upgradeability methods utilize \textit{Delegate call} opcode. Using  this opcode bring complexity to the system and needs more security considerations (\eg checking storage layout compatibility). Upgradeability methods that do not need to use delegate call opcode receive a check mark (\checkmark) otherwise it awarded nothing. 
 % JC: Not crazy about this column. It is not clear if it is a feature or a bug. I also don't really understand it. Forced to change it isn't clear. "Considerations to smart contract security" doesn't make sense to me. Try expanding on these ideas a bit or we can discuss in a call. 
 
%  Applying each upgradeability method has different complexity effects on the system. An upgradeability method in which the designer do not need to change any logic of the contract receives empty circle (\Circle). Upgradeability methods that cause changing the whole design or the logic of the smart contract but do not add possible security flaws or attack surfaces receive half circle (\LEFTcircle). Upgradeability methods that force the developer to change the whole logic of the smart contract and adds considerations to smart contract security receives full circle (\CIRCLE).

%  \textit{Migration} method does not add any complexity to the system. The system designer do not need to change considerable part of system design to have feature to migrate to the new version before deployment. So, it is the less complex choice that a designer can take (\Circle).
 
%  In \textit{Retail changes} method, it is hard and complex for system designers to design a system in which a parameter or a small part of system can change the logic of the system and be safe to change it but it does not add security considerations (\LEFTcircle).
 
%  In the \textit{Call-based} upgradeability approach, the system designer must separate storage and logic pieces of the contract and change the whole design of the smart contract. It adds complexity to the system to be confident that these two are separated correctly. But it does not add considerable new attack surfaces to the system (\LEFTcircle).
 
 
%  On the other hand, \textit{DelegateCall-based} patterns adds more complexity to the system because of using DelegateCall opcode as well as need to separating logic and storage contract. The main difference between \textit{Call-based} and \textit{DelegateCall-based} approach is that the developer should take care that using Delegatecall opcode needs to take care that the logic and storage contract must have similar storage layout. So using DelegateCall pattern changes the design of the contract and adds new attacks surfaces to the system (\CIRCLE).

 \subsection{No indirection}
 
 % JC: This is good. You can add more precision though. If I want to call function A, and I call B which calls A, it is called a "layer of indirection." Can this be phrases in terms of the number of layers of indirection for users? No indirection (full), one or more layers of indirection (empty).  
 
 Upgradeability methods that do not need any redirection receive a check mark (\checkmark).An upgradeability pattern that adds an extra gas because of adding one or more layers of indirection awarded nothing. An upgradeability method in which just a portion of its transactions need indirection receive square (\XBox). 

\textit{Retail Changes} and \textit{Migration} methods do not add any indirections to the system so will not alter the transaction costs (\checkmark). However, \textit{Call-based} and \textit{DelegateCall-based} approaches increase the transaction cost for users (receive nothing). \textit{Call-based} method will increase the transaction cost because whenever a transaction needs to add or change a data, the logic contract must \textbf{CALL} the storage contract which adds an extra gas. Also in \textit{DelegateCall-based} approach all transactions will be \textbf{Delegate Called} to the logic contract using a \textbf{proxy} which adds an extra gas on each transaction. 

\textcolor{red}{We need a comparison between Call based and Delegate call based approaches (if possible. I think Call based approach is gas efficient for systems that do not need to add or change data frequently. Also there are other extra adds-on that can add useability but adds gas like adding feature for uninformed users in Call based approaches.)}
% TODO: nice article: https://forum.openzeppelin.com/t/a-more-gas-efficient-upgradeable-proxy-by-not-using-storage/4111

%%%%%%%%%%%%%%%%%%%%%%%%%%%%%%%%%%%%%%%%%%%%%%%
%  \subsection{Cost of Upgrade}
%  % JC: This column merges two seperate ideas: do you have to deploy a new contract? That can be a column. Do you have to migrate the state from the old contract to the new contract? That can be a second column. 

%  One of the main differences between upgradeability approaches is how much does the upgrade process costs. An upgradeability pattern that costs like a normal transaction (\eg changing a parameter or a more complex variables like structures) receives an empty circle (\Circle). Upgradeability methods that needs to deploy a new logic contract but do not need to push old data into it receive half circle (\LEFTcircle) and patterns in which the upgrader should deploy a new logic contract and push old data into the new version is the most expensive approach in the upgrading process and awarded full circle (\CIRCLE).
 


%%%%%%%%%%%%%%%%%
%  \subsection{Useability}
%%%%%%%%%%%%%%%%%
% JC: This still vague. Be very precise. Alice uses DApp at address A before the upgrade. After upgrade, she can be unaware that upgrade happened. She uses address B instead. The function names have all changed. What else are "some actions in addition"? Give specific examples. (New ability can be added to upgrade which is not user friendly)

\subsection{User endpoint address not changed}
\textcolor{red}{The title could be: \textit{Having 2 Dapps at the end} ?!}
% Both are the same I think
%For instance in Uniswap we have 2 dapps at the end but can we say user should use a new address or not? maybe these 2 are not the same?
%which kind of users are we talking about? client users or like other smart contracts?!
In some upgradeability methods, after the upgrade process, users must call a new contract address to use the Dapp. For instance, Alice uses a DApp at address A before the upgrade. After upgrade, she can be unaware that upgrade happened or she may need to use address B instead (\checkmark).

%We may have another row for a call-based pattern that uses a "registry like patterns" to not change user end point

\subsection{Users do not require taking some action (withdraw deposit)}
\textcolor{red}{The title could be: \textit{Users are involved? (Client)} ?!}

In some upgradeability patterns, users are responsible to transfer their related data from the previous version to the new one. In some others, users do not need to take any further actions like this (\checkmark)). 
%Need to check for the Migration plans to find out if there is any other thing that users may do for the migration other than the withdraw deposit

%  An upgradeability approach in which users do not feel any changes and do not need to take any action after the upgrade process receives a full circle (\CIRCLE). In patterns that the users just need to call a new smart contract address rewarded a half circle (\LEFTcircle). In some patterns although the users should take some actions in addition to using a new address (\Circle).    
 
%  In \textit{Retail Change} approach the user does not feel the upgrade process unless she has a exposure to the parameter that is changing on the upgrade (\CIRCLE).
%  In \textit{Call-based} method, the logic contract (the contract that is called by users) will be changed. So, users must call a new contract after the upgrade which is not user friendly (\LEFTcircle). There are some approaches to mitigate this by using a proxy contract in between. The other way to mitigate this is to implement a way in which an old logic smart contract can call the newer version and pass the user's request to the newer version. These solutions add an extra gas to each transaction which is not usable as well.
% In \textit{DelegateCall-based} approach users are calling the proxy contract which delegate call their request to the whitelisted logic contract. In the upgrade event the developers change the whitelisted address to the new version but users call the proxy like before. So, users do not feel the upgrading process (\CIRCLE).

% On the other hand, users in a \textit{Migration} plan need to work with a new smart contract . In some Migration plans, users themselves must withdraw their fund from the previous version and deposit it on the new version which is not user friendly and costly for them (\Circle).

% %%%%%%%
%  \subsection{Fixing a Bug}
 %%%%%%%%%
%  \subsection{Remove this}

%  % JC: What is it about the contract that does or does not allow a bug to be fixed. Is it just a question of whether you can change the logic of the DApp? In that case, is it already captured by the proposed column above? 
 
% As mentioned in the previous part, upgradeability could be used in two different situations; adding new features or fixing a bug. An upgradeability pattern that cannot be used for fixing a bug receives square (\XBox). An upgradeability pattern which can be used for fixing a bug but not suitable for fixing a bug receives an empty circle (\Circle) and the methods which are suitable for fixing a bug receives a full circle (\CIRCLE).

% In an incident, \textit{Retail Changes} approach won't help to response to a bug or hack because the extent of upgradeability is limited and the developers are not able to change the required parts of the system (\XBox). \textit{Migration} is not proper as well because it is not quick enough to respond in a limited time window.
% On the other hand, \textit{Call-based} and \textit{DelegateCall-based} are very well for fixing a bug. In the event, the developers can find the root cause and patch it by deploying a new contract and change the implementation address in the logic contract. 

%%%%%%%%%%%%%%%%%%%%%%%%%%%%%%%%%%%%%%%%%%%%%%%%%%%%%%




\subsection{Change pattern after deployment}

\textcolor{red}{Migration is the only way to add upgradeability feature to out smart contract}

% This can be used for EOA, Multi-sig and governance



%TODO: Talk about the outcomes:
% for instance: 
% Usabality: this method is not usable because it changes the user endpoint address and need withdraw and deposit and bla bla ...
% Fixing a bug: ...
% Upgrade cost: .....
% Speed of upgrade: ....
% Level of decentralization: 


% \subsection{Pauseability}

% %%%%%%%%%%%%%%%%%%%%%%%%%%%%%%%%%%%%%%%%%%%%%%%%%%%%%%%%%%%%%%%%%%%%%%%%%%%%%%%%%%%%%%%%%%%%
%%%%%%%%%%%%%%%%%%%%%%%%%%%%%%%%%%%%%%%%%%%%%%%%%%%%%%%%%%%%%%%%%%%%%%%%%%%%%%%%%%%%%%%%%%%%

\subsection{Speed of an Upgrade}
 
 % JC: I would split this into speed of reaching a decision to upgrade, and speed to implement the upgrade once the decision is made. The second one maybe is already well-covered by previous columns though. I'm not worried about this and the next column this week. Let's work on the other ones first. 
 
Another difference between upgradeability methods is the speed in which an upgrade can be processed. This depends on the type of decision makers which will be discussed in~\ref{decisionMakers}. 

\textit{Retail changes} method is the fastest way to upgrade a system comparing to other methods. Using an EOA as the decision maker is the fastest option of an upgrade. Using multi-sig is a bit slower than using EOA. Utilizing a decentralized governance scheme to decide about upgrades will put an inherit time delay to the upgrades.

\textit{Migration} has the slowest upgrade process between other methods. The reasons are discussed in the previous parts (see~\ref{upgradeFRQ}).

\textit{Call-based} and \textit{DelegateCall-based} are very similar to each other in the speed of upgrade. These two are not as quick as \textit{Retail changes} because the developer needs to find the root cause of a bug or find the upgrades needed for system and then implement the smart contract and deploy it to the system which is time consuming.
On the other hand these two approaches are faster than \textit{Migration} because as mentioned before, in Migration plans we need to collect and push old data into newer version as well.  

 \subsection{Level of Decentralization}

The last and one of the most important characteristics that are different in upgradeability methods is the level of decentralization. An upgradeability methods that a single third party decides the upgrading process receives a square (\XBox). 
Using an \textbf{EOA} to decide about a change is the most central option that a system designer can choose regardless of the upgradeability method uses in the system. 
In case a group of whitelisted persons can decide on the changes orf the system using \textbf{Multi-sig} is not decentralized as well. Although it improves the level of decentralization of the system but at the end a specified number can decide to change the system. So it awarded an empty circle (\Circle).

Utilizing a decentralized governance model to vote for a change is a good way to make the decision making on the upgrades more decentralized. \textit{Retail Changes} using voting scheme is more decentral than \textit{Call-based} and \textit{DelegateCall-based} because boundary of changes are limited on the Retail methods so it awarded a full circle (\CIRCLE). But, in CAll-based and DelegateCall based methods the developers have the power to put some kind of backdoor in the system while upgrading and they receive a half circle (\LEFTcircle). 

The \textit{Migration} method is the most decentralized approach because it gives the users chance to decide whether to move to the newer version or not so it awarded a full circle (\CIRCLE). For instance, Uniswap uses this method for its upgrade and the users have choice to transfer their funds from Uniswap V2 to V3 or not and as we can see some users decide to stay on the previous version.

% • Efficiency: gas for those uninformed about upgrades
% • Efficiency: gas for those informed about upgrades

 % talk about vulnerabilities of each here
% Exploit on Aave caused by initialize function (Alternative to constructor) https://blog.trailofbits.com/2020/12/16/breaking-aave-upgradeability/

 \section{Upgrading process}
  

 \subsection{decision maker(s)} \label{decisionMakers}
 There is a debate on who is responsible for upgrading a Dapp. Different systems can choose one of these schemes to upgrade their Dapp depending on the complexity of the system, frequency of the changes needed for the system and how fast does the system need to upgrade in the incidents.
%History of changing from not having a single owner to then owner and then Multi-sig and then Decentralized voting
 \subsubsection{Externally owned Address}
The easiest and the fastest way to upgrade a system is through a single address which is the owner of smart contract. This is the most centralized solution we have for upgrading a system. The main problem with this issue is the security of the system because it only depends on a single private key hold by the owner. In case of malicious party or if an attacker find the owner's private or if the owner lose the key the entire system is on the risk.

First Dapps on the ethereum blockchain used this method for the upgrade but it is not used these days because it is far from the idea of \textit{Decentralization}. 

 \subsubsection{Multi-Sig}
 A \textit{m out of n} Multi-sig wallet is a smart contract that can manage a transactions only if m number out of a specified n EOAs agree and sign the transaction. We can use address of a Multi-sig wallet as the owner of the system. In case of a upgrade or responding to an incident m number of the governors can permit to upgrade the system.

 This is a better answer to the decision making of the upgrade compare to using an EOA in case of centrality while keeping the speed of an upgrade process. However, it is not decentralized. One way to reduce the level of centralization is to use different trusted teams who are stakeholders of the system in the multi-sig wallet. 


\subsubsection{Governance Voting}
The most decentralized way to decide on a system change is to do it using decentralized voting. This can be done by distributing governance voting tokens to the community and then they can vote on a change proposal by staking their voting token. 

There are some critique to this method. Governance by voting has an inherit time delay to the upgrading process. This raises a problem when the system needs an instant upgrade (\eg responding to an incident). This means we need another mechanism to quickly fix bugs and upgrade the system on the event of incidents in conjunction with the voting process (\eg Global shutdown in MakerDAO).

It is also not cost-efficient for the voters because all token holders must send a transaction and needs to pay network fee.

The other problem with this method is fair distribution of the tokens. If the governance token does not distribute fairly and the majority of tokens granted to the limited number of users, then it is very similar to the multi-sig method which is more costly and complex. Because, whales of the governance token can vote to any desired change of the system similar to the multi-sig.
 

\subsection{Mitigating risks}
There are critical setups on the systems to mitigate the possible risks on the upgrading process. We mention some of them here with risk associated with them.

\subsubsection{Timelocks}
In some project, there is a time window between every changes that approved on the system and when they affect the system. This gives opportunity to the users who are not satisfied with the upcoming upgrades to move their funds out of the system. However this is not proper in case of fixing a bug, because we need to patch the problem quickly.

\subsubsection{Thereshold}
In multi-sig and governance upgrade methods we need a threshold on votes to decide whether a change is approved or not. This threshold should be big enough to be confident that upgrading event represents the majority of opinions. On the other hand, the threshold shouldn't be that big because a big threshold will delay a system change. The system designer should consider that a portion of voters (signers in multi-sig or governance token holders in voting method) may not be available in the event of the upgrade and having a big threshold may result in halting the change proposal for a long period of time. In fact threshold has a trade-off between security/decentralization and speed of the upgrade process.

\subsubsection{Pausable}
In pauseable smart contracts, the decision makers (usually a multi-sig wallet) can freeze some or all operations of the system. Pausing a smart contract helps in some specific situations:

\begin{enumerate}
  \item Time to react to a bug or hack: usually it takes time to analyze and find the reason of a hack and patch the bug. In this time period the core developer team needs to pause the system to stop attacker from draining all the fund.
  \item Halting system in the upgrade process: For instance, in an ERC20 token contract upgrade we need to pause the system to stop users from transferring tokens during the upgrade. 
  \item Inactivating the previous version of the logic contracts: After an upgrade we need to have a plan to stop users from using the previous logic contracts. One way to do so is to make the logic contracts pauseable and pause them after the upgrade. 
\end{enumerate}

\subsubsection{Escape Hatches}
A escape hatch is a mechanism that lets the users to move their fund out of the system in the pausing events. For instance, in MakerDAO we have an emergency shutdown mechanism that pauses the system in the black swan events. But, users have the ability to extract their funds out of the system while the system is paused.
\subsubsection{Front-Runnign}
Upgrading a smart contract can be done by sending a transaction into the system. If the upgrade is a response to a unknown bug, then the upgrade process will hint attackers who is listening to the mempool to find the bug and hack the smart contract just before the upgrade. So there should be some mechanisms to mitigate front-running attacks. One solution to this issue is to use commit-reveal schemes. The team first sends a commitment of the upgrade (hash of the upgrade) to the system and after the timelock they can push and apply the original code which cannot be front run. 
%Commit reveal



% upgrading in dark: https://forum.makerdao.com/t/mip15-dark-spell-mechanism/2578


 \section{Measurement study}
% Level of intervention of each Dapp (pauseable etc.)
% History of upgrades --> How many times a Dapp upgraded before (using events etc...)

% \section{Concluding Remarks}


% \subsection{Registry pattern}
% Registry contracts are probably the simplest approach to upgradeability. Registry pattern consist of two main contracts: registry and logic contract. Registry contract holds the addresses of logic contracts and whenever it receives a transaction it will pass it to the related logic contract. 
% If the development team decide to upgrade the smart contract, they can deploy another smart contract and then just change the pointer of the registry smart contact to the new smart contract.

% The main disadvantage of this approach is that in upgrading event, there should be a manual or automated migration plan to transfer data from the old contract into the new upgrade smart contract.
% Another drawback of this pattern is that it also introduces additional complexity for external clients who would also need to call into the registry before interacting with the system


%Idea: see how frequent each project uses its upgradeability feature to upgrade the contract and why?

%In proxy storage layout you should take care that: 
% 1. never remove variable
% 2. never change var type
% 3. never change inheritance order


%Implementation of Eternal storage in call based upgrades: https://medium.com/cardstack/upgradable-contracts-in-solidity-d5af87f0f913
% Good classificaton: https://medium.com/1milliondevs/solidity-storage-layout-for-proxy-contracts-and-diamonds-c4f009b6903
% New storage type *Dimond Storage*: https://medium.com/1milliondevs/new-storage-layout-for-proxy-contracts-and-diamonds-98d01d0eadb
%https://medium.com/1milliondevs/new-storage-layout-for-proxy-contracts-and-diamonds-98d01d0eadb
% Upgradability checklist: https://blog.trailofbits.com/2020/06/12/upgradeable-contracts-made-safer-with-crytic/



%ToDo: Write about OpenZeppling upgrade bug : https://blog.trailofbits.com/2018/09/05/contract-upgrade-anti-patterns/
%ToDo: Gas cost comparison between different implementation (using different number and types of the storage variables)


\section{discussion}
%\subsection{Off chain upgrades (UNiswap Arbitrum)}

% !TEX root = ../main.tex

\begin{table}[t!]
    \centering
    
        \begin{tabular}{lllllllllllllll}
    
    &
    \headrow{Can replace entire logic} &
    
    \headrow{can replace pre-specified part of logic} & 


    \headrow{Can replace entire state} &
    
    \headrow{can change pre-specified state variables} &
    
    \headrow{No need to deploy a new contract} &

    \headrow{No need to migrate state from old contract} &

    \headrow{No need to separate State and Logic} &

    \headrow{Function Selector Clashes Risk} &

    \headrow{Storage Clashes Risk} & 

    \headrow{No indirection} & 


    \headrow{User endpoint address not changed} &
    

    \headrow{Downtime in upgrade events} &

    \headrow{No need to change code to add the upgrade pattern} &

    \headrow{Need to change a state variable} 
    
    
    \\
    
    \hline
 
    
        \multicolumn{1}{c|}{Parameter Configuration}	& \multicolumn{1}{c|}{}  & \multicolumn{1}{c|}{} &  \multicolumn{1}{c|}{} & \multicolumn{1}{c|}{\checkmark} & \multicolumn{1}{c|}{\checkmark} & \multicolumn{1}{c|}{\checkmark} &  \multicolumn{1}{c|}{\checkmark} &  \multicolumn{1}{c|}{} & \multicolumn{1}{c|}{} & \multicolumn{1}{c|}{\checkmark} & \multicolumn{1}{c|}{\checkmark} & \multicolumn{1}{c|}{} &\multicolumn{1}{c|}{\checkmark} & \multicolumn{1}{c}{\checkmark}\\
    
        \hline
  
        \multicolumn{1}{c|}{Component Change}	& \multicolumn{1}{c|}{}  & \multicolumn{1}{c|}{\checkmark} &  \multicolumn{1}{c|}{} & \multicolumn{1}{c|}{} & \multicolumn{1}{c|}{} & \multicolumn{1}{c|}{\checkmark} &  \multicolumn{1}{c|}{\checkmark} &  \multicolumn{1}{c|}{} &  \multicolumn{1}{c|}{} &  \multicolumn{1}{c|}{\XBox} & \multicolumn{1}{c|}{\checkmark} & \multicolumn{1}{c|}{} & \multicolumn{1}{c|}{\XBox} &  \multicolumn{1}{c}{\checkmark}\\
        

        \hline

        \makecell{Contract Migration}	& \multicolumn{1}{|c|}{\checkmark}  & \multicolumn{1}{c|}{} &  \multicolumn{1}{c|}{\checkmark} & \multicolumn{1}{c|}{} & \multicolumn{1}{c|}{} & \multicolumn{1}{c|}{} &  \multicolumn{1}{c|}{\checkmark} &  \multicolumn{1}{c|}{} &  \multicolumn{1}{c|}{} &  \multicolumn{1}{c|}{\checkmark} & \multicolumn{1}{c|}{} & \multicolumn{1}{c|}{} & \multicolumn{1}{c|}{\checkmark} &  \multicolumn{1}{c}{}\\
    
         \hline
         \makecell{Create2 metamorphosis}	& \multicolumn{1}{|c|}{\checkmark}  & \multicolumn{1}{c|}{} &  \multicolumn{1}{c|}{\checkmark} & \multicolumn{1}{c|}{} & \multicolumn{1}{c|}{} & \multicolumn{1}{c|}{} &  \multicolumn{1}{c|}{\checkmark} &  \multicolumn{1}{c|}{} & \multicolumn{1}{c|}{}&  \multicolumn{1}{c|}{\checkmark} & \multicolumn{1}{c|}{\checkmark} & \multicolumn{1}{c|}{\checkmark} & \multicolumn{1}{c|}{\checkmark} &  \multicolumn{1}{c}{}\\

    
        \hline
        \makecell{Consensus Override}	& \multicolumn{1}{|c|}{\checkmark}  & \multicolumn{1}{c|}{} &  \multicolumn{1}{c|}{\checkmark} & \multicolumn{1}{c|}{} & \multicolumn{1}{c|}{} & \multicolumn{1}{c|}{} &  \multicolumn{1}{c|}{\checkmark} &  \multicolumn{1}{c|}{} & \multicolumn{1}{c|}{}&  \multicolumn{1}{c|}{\checkmark} & \multicolumn{1}{c|}{\checkmark} & \multicolumn{1}{c|}{\checkmark} & \multicolumn{1}{c|}{\checkmark} &  \multicolumn{1}{c}{}\\

        \hline

        \makecell{Call-based}	& \multicolumn{1}{|c|}{\checkmark}  & \multicolumn{1}{c|}{} &  \multicolumn{1}{c|}{} & \multicolumn{1}{c|}{} & \multicolumn{1}{c|}{} & \multicolumn{1}{c|}{\checkmark} &  \multicolumn{1}{c|}{} &  \multicolumn{1}{c|}{} &  \multicolumn{1}{c|}{} &  \multicolumn{1}{c|}{} & \multicolumn{1}{c|}{} & \multicolumn{1}{c|}{} & \multicolumn{1}{c|}{} & \multicolumn{1}{c}{\checkmark}\\        
        
         \hline

         \makecell{DelegateCall-based}	& \multicolumn{1}{|c|}{\checkmark}  & \multicolumn{1}{c|}{} &  \multicolumn{1}{c|}{} & \multicolumn{1}{c|}{} & \multicolumn{1}{c|}{} & \multicolumn{1}{c|}{\checkmark} &  \multicolumn{1}{c|}{} &  \multicolumn{1}{c|}{\checkmark} & \multicolumn{1}{c|}{\checkmark}&  \multicolumn{1}{c|}{} & \multicolumn{1}{c|}{\checkmark} & \multicolumn{1}{c|}{} & \multicolumn{1}{c|}{\XBox} & \multicolumn{1}{c}{\checkmark}\\    
        
         \hline
        
        
        \end{tabular}
        \captionsetup[tabular]{singlelinecheck=off}
        \caption{Evaluation\ref{tab:eval}}
       
    
    \end{table}
    \footnotetext[1]{Design of system in which a parameter can change the logic is hard}

%        \makecell{Diamonds}	& \multicolumn{1}{|c|}{\checkmark}  & \multicolumn{1}{c|}{} &  \multicolumn{1}{c|}{} & \multicolumn{1}{c|}{} & \multicolumn{1}{c|}{} & \multicolumn{1}{c|}{\checkmark} &  \multicolumn{1}{c|}{} &  \multicolumn{1}{c|}{\checkmark} & \multicolumn{1}{c|}{\checkmark\checkmark}&  \multicolumn{1}{c|}{} & \multicolumn{1}{c|}{\checkmark} & \multicolumn{1}{c|}{} & \multicolumn{1}{c|}{\XBox} & \multicolumn{1}{c}{\checkmark}\\

%



\section{What is RB tokens}
%%"A currency, to be perfect, should be absolutely invariable in value," said David Ricardo in 1817.
%%There is a huge debate about the charectaristics of a stable asset. One may claim that a stable asset or currency is an asset that helps its owner to have stable purchasing power. Others may claim that an stable asset should have invariable quantity.

There is a growing tendency to issue asset on top of blockchain that represents  real world assets such as shares, commodities, and currencies. 

One way to implement this kind of tokens is that a company obtains a reserve of the asset and issue tokens that represents a unit of asset. But this design needs custodianship proofs, periodic audits and also trust on the third party.

The question raised here is whether it is possible to find a solution to remove the trust on the third party?

The answer is RB token Dapp. There are two parties involving on each contract of the system. The Red token holder who need a representation of the underlying asset on the blockchain, and Black token holder who bets against the pair value of ETH and the underlying asset.

Therefore, the amount of deposited ETH on each new agreement will grow or shrink depending on the exchange rates of the underlying asset and ETH. Because a blockchain has no inherent knowledge of exchange rates, this mechanism still requires one trustworthy entity called an oracle to write the exchange rates into the blockchain (or consensus can be taken across a set of oracles).

\emph{Working Example}: Assume Alice wants to create representation of Google share (GOOGL). She sets up a DApp that can hold ETH and issue tokens. The DApp determines how much ETH is equivalent to 1.5 GOGGL, using the current exchange rates, provided to the DApp by a trusted third-party oracle, and Alice deposits this amount of ETH into the DApp. The DApp issues to Alice two tokens, Red and Black. At some future time, the holder of Red token can redeem up to equivalent value of 1 GOOGL in ETH from the deposit and the holder of the Black token gets any remaining ETH. Alice will transfer the Black token to Bob who wants to bet against ETH/GOOGL. When Alice redeems the Red token, it will be worth 1 GOOGL in ETH when the entire deposit of ETH is worth more than 1 GOOGL. If the exchange rate of ETH drops enough or the exchange rate of the GOOGL raise enough (or combinatoion of them), the entire deposit will be worth less than a 1 GOOGL—Alice will get all of the deposit, and the holder of the Black token will get nothing.

There are two risks on the system: Volatility risk of ETH and the underlying asset. Decision on the system depends on the spot exchange rate of ETH and Underlying asset
($\frac{P_{ETH}}{P_{GOOGL}} $).


\section{Analysis}



\section{Systemization}
There are a number of stablecoins using crypto-assets as collateral to issue stablecoins which shows a very broad design landscape for indirectly-backed stable coins and different design goals and strategies behind stable coin systems. In this part we will discuss about the mechanism that could be added on top of the RBcoin system discussed on previous section that can be used to change the propetries of the system. The designer of a stable coin may have design goals like Fungibility(I think it should be removed)(Money like tokens could be replaced or sth like that), Stability, Simplicity, and Decentrality. 

Firstly, the designer sets the goals of the design and then assign the design parameters of the system to acheive the design goals.

In this part, we propose a systematical design-decision model for indirectly-backed stablecoins. The designer is facing four main design parameters to create a new indictly-backed stable coins which are Maturity date, Counter-party, Collateral risk and interventions. 



An overview of the indirectly-backed stablecoins design landscape is in Figure\ref{land}.

\begin{figure} 
\centering
\includegraphics[width=12cm]{Mindmap}
\caption{overview of the indirectly-backed stablecoins design landscape}
\label{land}
\end{figure}

\subsection{Maturity Date}
The indirectly backed stable systems are agreements between two different parties, a stable party and a volatile party. These two parties should be different because holding both sides of the contract is equal to keeping the underlying asset on the wallet and it is not logical to participate on such a system to just keep the asset.

In this agreement the parties agree to split their deposited asset at a specified date named maturity date. At the maturity date the stable party will receive an exact amount of underlying asset based on the price of the asset at the settlement date and the volitile party will receive the remained part.

The first parameter that the designer should decide about is the maturity date. The maturity could be happened at a fixed specific time or could be perpetual.
\subsubsection{Fixed Maturity dates}
The designer can set specific dates for the contracts to be matured for example at the first day of each months. At the day of maturity the deposited assets are devided into two parts. The \$1 equivalent amount of the asset goes to the pocket of the stable party (if possible) and the remained is for volatile party. This is simlar to futures contract in Finance.

The designer may let one of the parties to exercise the contract beforr the maturity date. It is very similar to american options. The exerciser could be the black coin holder, red coin holder or both of them. 

\subsubsection{Perpetual contracts}
The other way to design the stable coin contracts is to make them as a perpetuty. In this mechanism there is no specific day for the parties to mature the contract. The system may have the option of settlement.

The perpetual stable coin without settlement is not rational because it is like burning the underlying asset to issue stable coins. 

For open-ended perpetual stable coins one of the parties has right to request for settlement whenever she wants. The exerciser could be black coin holder or red coin holder.
\subsection{Counter party}
One of the design parameters is to decide who is the risk taker in each contract. Each newly issued contract is backed by a different vault. Each black coin points its vault because the amount of deposits depend on the black coin holder decision. 
The designer could decide whether use a pool for red coins or pair each red coin to its related black coin.

\subsubsection{Pooled}

In this type of design the black coins are pointing to the vaults. Red coins are fungible on the system and there is no differences between them. There is no relations between red coin and the black coin. 

In this class, all red coin holders are risk takers of the system and on the unexpected events the take the risks regardless of their issuance contract.

In this case a black coin holder do not know the counterparty and the risk is pooled here.

\subsubsection{Paired}

The designer can create a system in which each red coin and black coins are paired and point to the issued contract. Now the black coin holders now their counterparty and the red coin holder is the risk taker of its own vault. So the risk is not pooled here.
\subsection{Collateral Risk}
If the price of the underlying asset goes down, the value of deposit falls below collateral ratio. The designers have the choice to react to this situation. We described some of different possible decisions:

\paragraph{Liquidation}
This variety of design is similar to the marginal accounts in traditional finance. If the underlying asset decreases in price and the value of a vault runs under the collateralization ratio, the system will liquidate the vault.

Liquidation occurs in the case that a vault is the danger area and may not able to pay its obligation. The deposited assets will transfer to the person who takes the responsibility of the debt of the vault. In other words, the liquidator should pay the borrowed red coin (and other obligations in a type of design, such as stability fee in MakerDAO) and receive the vault in exchange.


In liquidation design, the majority of vaults are over-collateralized. If a vault goes under-collateralize it will be liquidated. So, the red coin holders are pretty sure that their coin is backed by a dollar (. They can exchange their asset because the red coins are similar.

Smart contracts cannot trigger themselves. So, there should be an outsider player (like Keepers in MakerDao) to liquidate the warned vaults. They should trace the system to find the under-collateralized vaults and call the liquidation function. Then the keeper will transfer the debt of the vault to the smart contract and receive the deposit of the vault in exchange.

There are design parameters in the liquidation method:

\begin{enumerate}
  \item Collateralization Ratio:
This number shows the factor of over-collateralization. The ratio between the value of the deposits on each vault and the value of borrowed red coins should always be more than the collateralization ratio.
  
The collateralization ratio is depending on the volatility of the underlying asset. For instance, in the MakerDAO platform, the collateralization ratio for ETH vaults is 1.5. However, in the Synthetix project, it is 7.5 for SNX token, which is more volatile than ETH.
  \item Liquidation Incentive:
A smart contract is not able to trigger itself. An outsider player named liquidator should pay the transaction fee to call the liquidation function of the smart contract. Some factors influence the costs and profit of liquidators:
\begin{enumerate}
	\item Transaction fee:
The liquidator should trigger the liquidation function, send the sufficient red coins, and bid on the auction to win the vault. The user has to pay the transaction fee for these processes. The transaction fee depends on the time of transaction and network congestion.  
	\item Cost of capital: The liquidator pays the obligation by Red coin and receives the deposited coins of the vault. Therefore, the liquidator needs a sufficient amount of red coins for liquidation. There is an opportunity cost associated with the decision of the liquidator to not lend her capital and gain interest.

In other scenarios, the liquidator may just borrow the fund from lending platforms to liquidate the vault and pay back the loan afterward. There is a cost of borrowing in this scenario as well. So, there is a cost of capital for the liquidator.
	\item Price Oracles: RB coin systems need the price of the underlying asset in USD. Blockchains have no access to externals data. Oracles are outsider systems that collect the price and push them to the blockchains. Price inefficiency may impose extra costs on the liquidator. For instance, if the price of the asset is \$100 on the markets, although the oracle price is \$90, the liquidator spends more to provide liquidity from the markets.
\end{enumerate}  

The designer has to incentivize the liquidators to trace the blockchain, find alerted positions, and then send transactions to liquidate them. 

The mechanism of the incentivization is varied between protocols. A majority of platforms give the liquidator discounts on the vaults. For instance, in Single Collateral DAI (SAI), there was a \%3 discount on the liquidation process. Other platforms are using auction models to let the market decide about the value of the vault. 
  
  \item Auction model:
In the case of liquidations,  liquidators may come up with a specific warned position and want to liquidate it. The system designer has different options for the decision of picking the winner liquidator.
 
The simplest implementation mechanism is the First Come First Serve. However, it won't be fair for the vault holder if the first liquidator bids with a low amount of red coins.

There are other auction-based mechanisms to find the liquidator. The question raised here is, which method is the most efficient and fair auction model for both bidders and vault holders.

MakerDao utilizes a mixture of an absolute-auction and a reverse-auction model for the liquidation process.

The absolute auction model is used until the bids cover the debt of the vault. When the bids pass the debt, the auction reversed, the bidders bid on a lower amount of the deposits on the vault for a specific amount of DAI tokens, specified on the absolute auction step.
  
  \item Liquidation penalty:
The liquidation penalty is an extra punishment for the black coin holders to care about their debt to collateral ratio.
MakerDAO platform charges liquidated vault extra \%13 as a punishment for their vault holders. There are two main reasons to add Liquidation penalty on the design:

\begin{enumerate}

  \item To force vault keepers to be over-collateralized
  \item To mitigate grinding attacks: grinding attack occurs when the position holder deliberately unsafe her black coin and participate in the liquidation auction against her position to buy her deposited assets cheaper.

\end{enumerate}
\end{enumerate}

Using liquidation mechanism as a shield to protect the system provoke criticism. Possibly the black coin holders are freaking out when the price of ETH drops significantly. They may proceed to net out just before the liquidation occurs to their vaults. We called this situation "early liquidation" of the system. 

In this case, the system needs more ETHs to be deposited. The liquidation mechanism is designed to force the black coin holders to inject more ETHs to the system, but the black coin holders withdraw their deposits, and the total collateral of the system drops significantly, which is not the goal of the designers in this situation.

\paragraph{Withholding rewards}

Majority types of designs like liquidation are disincentivizing bad actors. But, another approach is to encourage users to act properly and incentivizing good actors of the system.

For instance, in the Synthetix project users need collateralize SNX tokens (Synthetix network token) to receive sUSD tokens (Synthetix stable coin pegging a USD). There is no margin call methods or liquidation function on the design. But, there is a reward on the system for users who keep more than the over-collateralization ratio. 
The Synthetix system has \%2 annual inflation on SNX tokens. The inflationary tokens are allocated to the vaults that hold more than the collateralization ratio.

There is another reward for the system. The traders on the exchange of the Synthetix project, pay transfer fees collected and distributed to the vaults holding more than the collateral ratio.

The system incentivizes people to stake their SNX token to be over-collateralized and receive the rewards. So, there is no punishment in the system for bad actors in this class of design.

\paragraph{Banning Black coin holders}

In the design of indirectly-backed stable coins, the red coins are not redeemable. In other words, the red coin holders cannot give back their red coins to receive \$1 of the deposited ETH in exchange. The red coin holder must own (or buy if possible) a black coin to net out a vault and receive the ETH.

In the liquidation scenario, the designer pushes black coin holders to be over-collateralized, applying liquidation punishments. Red coin holders and arbitragers are confident that there is no difference between the red coins because each of the red coins is backed by a sufficient amount of ETHs to be \$1 (with high probability).

In another class, the designer removes the liquidation mechanism, prevent black coin holders from withdrawal. In the fungible black coin design class, the designer also forbids the black coin holders from transferring their token. 

In this situation, the incentives for black coin holders to be over-collateralized has been decreased, compared to liquidation design class. But, there is a huge incentive left for them to be over-collateralized. If the price of the underlying asset drops, the black coin holders may want to sell their deposited assets to reduce the loss. In this scenario, just black coin holders that have over-collateralized vaults can net out and receive their underlying assets to sell them to the market.

This type of design will increase the fluctuation of the price of the red coin. The market watches the aggregated collaterals on the system and the number of red coins issued by the system and also the price of the underlying asset to evaluate the price of red coins. So when the price of underlying asset drops, the price of red coins will reduce concerning the underlying asset price. 

In this scenario, red coin holders are taking parts of the risk of underlying asset volatility risk. On the situation that the price of the asset drops significantly, the value of the stable coin will fall.

\paragraph{Reputation systems}

In traditional finance, reputation scoring systems are used to decrease or eliminate the collateral needs for a specific financial transaction. Participants are utilizing their reputation as collateral or source of trust for financial services. 

For instance, in the FICO credit score system, users can enhance their credit limit by increasing their credit score. There is a default risk on credit systems, but the defaulted person will be punished by credit score reduction. The bad actor loses reputation scores forbidden from using plenty of financial services. Therefore, users have adequate incentives to pay their bills.

A revolution of decentralizing the finance products on top of blockchain technologies began in early 2018, named Decentralized Finance (Defi) movement. There is a myriad of different decentralized financial services out there, such as MakerDAO, Compound, Synthetix, Aave, etc. 

There are no differences between users that act properly on Defi platforms and the bad actors. The DeFi ecosystem suffers from a lack of a reputation system or reputation scoring. Using a reputation system will incentivize users to act properly and also reduce the default risk of the system. On the other side of the coin, the users with high-grade reputation scores have new opportunities. So, the cost of defaulting will be increased for high-grade users.

There are barriers to implementing an effective reputation system on blockchains. Lack of strong identities or anonymity is one of them. Also, users can create fake histories. However, these are not impossible to address.

In case that our system concludes a trustworthy reputation system, the designer can use reputation as collateral.  We describe two different designs using reputation systems:
\begin{enumerate}
	\item Reputation-based collateral ratio: 
In the design of the system, the collateral ratio could be reliant on the reputation of the user. In other words, the collateral ratio is higher for new users (users with no reputation) and lower for users that act properly for a long time.
	\item Reputation-based staibility fee
In systems like MakerDAO, the DAI borrowers are obliged to pay a fee on their borrowing named stability fee. This stability fee is being set by Maker token holders.
	In a design based on the reputation, the stability fee could be dependent on the reputation of the user. The reputable user is paying a lower stability fee compared to the new users.
\end{enumerate}
 
\paragraph{Insurance}
Insurance models are used to hedge the risk of unexpected events in different systems. Under-collateralization of a vault is an unexpected event on the RBcoin system. The designer could use an insurance model to protect parties from financial loss in the case of under-collateralization. 

On the insurance model, the insurer will pay a premium to the insurance company. The company will protect the client from financial loss. 

In RBcoins there could be a built-in or outsourced insurance model to protect parties from under collateralization loss. The question raised here is who should pay the premium.

\begin{enumerate}
	\item Premium pay by Red coin holders:
It is very similar to Credit Default Swaps (CDS) on traditional finance. In this design, the approach is that the red coin holders are lending some amount of money to black coin holders. Therefore, black coin holders are borrowing from red coin holders to have a leveraged position on the underlying asset. In this situation, the red coin holders can pay an insurance premium to the contract to protect themselves from the default risk of black coin holders. In the case of under-collateralization, if the black coin holder cannot afford the loss, the insurance contract will pay the loss to the red coin holder.

This type of design is implemented on the MakerDAO platform. The DAI borrowers are paying a premium so-called stability fee to the system. These fees are collected on a pool named Maker Buffer pool. In the case of liquidation of a CDP, if the winner of the auction pays a lower amount of the obligation of the vault, the difference between the obligation and the paid amount will be paid by the Maker Buffer pool.

	\item Premium pay by Black coin holders
In this type of design, the black coin holders are paying the insurance premium. It is similar to regular insurance contracts in which the insurer buys an asset and guarantee it by paying a premium to insurance companies. For example, a person purchases a house and insure it.
Here the black coin holders are buying a position and pay the insurance premium. If the price of underlying asset drops and the vault going to be liquidated the insurance contract will pay on behalf of the insurer.
	\item Premium pay by both
In this scenario, both Red and Black coin holders are paying the insurance premium to ensure their positions. 
\end{enumerate}
\subsection{Intervention}

The designer should indicate the level of intervention on the system. There could be some mechanisms to change the level of properties on the system. For instance, the designer may use interest rate models, algorithmic models or secondary tokens to acheive a system property.

\subsubsection{Float}

The desiner may let the system to be free of human or algorithm intervenions. This is the most decentral type of design. This type of design benefits simlicity and decentrality but the red coins has fluctuations.

\subsubsection{Interest rate models}

Interest rates are tools that a system governer has to stimulate the demand or supply side of a system. The designer of indirectly-backed stable coins can use interest rate models on red coins or black coins (two different rates) to adjust the supply and demand of black coins and red coins in the system. This rates could be adjusted by human intervention like DAI or could be fully algorithmic.

\subsubsection{Parameter setting}

All types of the indirectly-backed stable coin system has some parameters in common such as collateral ratio. In specific designs, there are other design parameters added to the system such as maturity dates, insurance premium and rates. The designer may decide to intervene on the system and change these parameter over time. This intervention could be by humans or algorithmic. 

\subsubsection{Issue haircut}

The designer should indicate that in case that the price of underlying asset goes down, who is the risk taker of the system. On regular system risk taker of the system is black coin holders till the price falls strongly and the black coins worth zero, then the red coin holders take all the risk.

The designer has other option to divide the risk between red coin holders and black coin holders. The designer could set hair cut parameter a (< 1). It means that if the price of the underlying asset falls 1 percent the redcoin holder now takes a percent of the risk and the the black coin holder takes 1-a percent.

This haircut parameter could be changed during the time by human intervention or algorithmic.


\subsubsection{Algorithmic rebasing models}

The designer may use rebasing models to change the number of tokens in a way that at the end of each day the price of redcoins is exactly a dollar.

%\section{Design Landscape}
The purpose of designing a crypto-asset backed stablecoin is to create a stable asset out of a volatile asset. There are a number of stablecoins using crypto-assets as collateral to issue stablecoins which shows a very broad design landscape for indirectly-backed stable coins.

In this part, we propose a systematical design-decision model for indirectly-backed stablecoins. There are some key design decisions for each core feature. We describe core features and the key design decisions related to the main features, and also we will discuss the pros and cons of each feature.

An overview of the indirectly-backed stablecoins design landscape is in Figure \ref{land}.

\begin{figure*} [ht]
\centering
\includegraphics[width=16cm]{Mindmap}
\caption{overview of the indirectly-backed stablecoins design landscape}
\label{land}
\end{figure*}

\section{Fungibility}
Fungibility or Interchangeability refers to the feature of an asset to be exchanged by the same asset type with equal quality and quantity. For instance, dollar bills are fungible because people can exchange the same amount of dollars without any frictions.

The first design decision is whether the tokens on the systems should be fungible or should be non-fungible. 

Fungiblity decision is devided into two parts:
\begin{itemize}
  \item Red coins Fungibility
  \item Black coins Fungibility
\end{itemize} 
We will discuss them seperately on next sections.


\subsection{Red coins Fungibility}
The Red coins, the stable coin of the system, could be fungible or non-fungible. All currently implemented indirectly-backed stablecoins are using fungible stablecoin design. However, it could be non-fungible as well.

\subsubsection{Non-fungible Red coins}
A stablecoin system designer could allow red coins to be non-fungible. For instance, each token could be backed by a different amount of ETHs without limitations on the system such as collateral ratio, liquidation and etc.

In this design class, the Red coin and Black coin should be pair-wise. Because each pair is backing by a different vault and a specific amount of deposited ETH.

There should be a system parameter for each red coin, depends on its vault to distinct the coins.
For example, each red coin could be marked by debt to collateral ratio, the number of minted red coins divided by the value of deposited ETH on the related vault, which clarifies the safeties of the coin.
So the buyer could have speculation on the price of each red coin base on the debt to collateral ratio.

Some reasons push designers to make red coins fungible. The first reason is usability. Assume that Alice wants to buy 100 red coins. On the other side, Bob wants to sell just 30 coins, Carol wants to sell 50 and David wants to sell 20 red coins. Now Alice should make a price speculation of three different coins and buy them at different prices which is not convenient for her.

The other issue with the non-fungible red coins is price discovery. Markets and crowd wisdom help to aggregate different opinions on the value of an asset. The aggregated results will discover the efficient price of the asset. In non-fungible design, there is not a straight relation between the price of each red coin and the specific characteristic of them like the debt to collateral ratio. So, Each person has speculation for each coin. Because of non-fungibility, the aggregations would not happen and so the real price will not be discovered.

The other reason is that stable coin users are willing to use the stable coins as a money. So, they need stable coins serve as a unit of account which means that you can price other goods or assets using the stable coin. In non-fungible design each red coin has a specific price. So, users can not price other goods based on the stable coin.


\subsection{Fungible Red coins}
As mentioned in the previous part, stable coin designers put all their effort to create stable coin systems including fungible red coins. There are different mechanisms to bring fungibility into red coins. We classify them into two key designs discussed in the next sections.

\begin{itemize}
  \item Under-collaterallization
  \item Separate maturity dates
\end{itemize} 


\subsubsection{Collateral Value Assurance}

One of the methods to bring fungibility into red coins is to set a lower limit for vaults (Collateralization ratio). If the value of deposited ETH in a vault drops beneath a specific amount, then the system decides to take an action. 

In this circumstance, there is a minimum amount of ETHs (collateral ratio)  backing each red coin. It secures the fungibility of red coins. Red coin holders are sure that there is at least a dollar in the vault for their red coin  (strongly expected). There is no difference between their red coin and the red coin of other people in this circumstance.

We will discuss various designs that differ on how they secure the vaults from the under-collateralization in the next sections. Some of them are incentivizing vault keepers to keep their deposit more than collateralization ratio and some of them are disincentivizing the bad actors of the system.
 
\paragraph{Liquidation}
This variety of design is similar to the marginal accounts in traditional finance. If the underlying asset decreases in price and the value of a vault runs under the collateralization ratio, the system will liquidate the vault.

Liquidation occurs in the case that a vault is the danger area and may not able to pay its obligation. The deposited assets will transfer to the person who takes the responsibility of the debt of the vault. In other words, the liquidator should pay the borrowed red coin (and other obligations in a type of design, such as stability fee in MakerDAO) and receive the vault in exchange.


In liquidation design, the majority of vaults are over-collateralized. If a vault goes under-collateralize it will be liquidated. So, the red coin holders are pretty sure that their coin is backed by a dollar (. They can exchange their asset because the red coins are similar.

Smart contracts cannot trigger themselves. So, there should be an outsider player (like Keepers in MakerDao) to liquidate the warned vaults. They should trace the system to find the under-collateralized vaults and call the liquidation function. Then the keeper will transfer the debt of the vault to the smart contract and receive the deposit of the vault in exchange.

There are design parameters in the liquidation method:

\begin{enumerate}
  \item Collateralization Ratio:
This number shows the factor of over-collateralization. The ratio between the value of the deposits on each vault and the value of borrowed red coins should always be more than the collateralization ratio.
  
The collateralization ratio is depending on the volatility of the underlying asset. For instance, in the MakerDAO platform, the collateralization ratio for ETH vaults is 1.5. However, in the Synthetix project, it is 7.5 for SNX token, which is more volatile than ETH.
  \item Liquidation Incentive:
A smart contract is not able to trigger itself. An outsider player named liquidator should pay the transaction fee to call the liquidation function of the smart contract. Some factors influence the costs and profit of liquidators:
\begin{enumerate}
	\item Transaction fee:
The liquidator should trigger the liquidation function, send the sufficient red coins, and bid on the auction to win the vault. The user has to pay the transaction fee for these processes. The transaction fee depends on the time of transaction and network congestion.  
	\item Cost of capital: The liquidator pays the obligation by Red coin and receives the deposited coins of the vault. Therefore, the liquidator needs a sufficient amount of red coins for liquidation. There is an opportunity cost associated with the decision of the liquidator to not lend her capital and gain interest.

In other scenarios, the liquidator may just borrow the fund from lending platforms to liquidate the vault and pay back the loan afterward. There is a cost of borrowing in this scenario as well. So, there is a cost of capital for the liquidator.
	\item Price Oracles: RB coin systems need the price of the underlying asset in USD. Blockchains have no access to externals data. Oracles are outsider systems that collect the price and push them to the blockchains. Price inefficiency may impose extra costs on the liquidator. For instance, if the price of the asset is \$100 on the markets, although the oracle price is \$90, the liquidator spends more to provide liquidity from the markets.
\end{enumerate}  

The designer has to incentivize the liquidators to trace the blockchain, find alerted positions, and then send transactions to liquidate them. 

The mechanism of the incentivization is varied between protocols. A majority of platforms give the liquidator discounts on the vaults. For instance, in Single Collateral DAI (SAI), there was a \%3 discount on the liquidation process. Other platforms are using auction models to let the market decide about the value of the vault. 
  
  \item Auction model:
In the case of liquidations,  liquidators may come up with a specific warned position and want to liquidate it. The system designer has different options for the decision of picking the winner liquidator.
 
The simplest implementation mechanism is the First Come First Serve. However, it won't be fair for the vault holder if the first liquidator bids with a low amount of red coins.

There are other auction-based mechanisms to find the liquidator. The question raised here is, which method is the most efficient and fair auction model for both bidders and vault holders.

MakerDao utilizes a mixture of an absolute-auction and a reverse-auction model for the liquidation process.

The absolute auction model is used until the bids cover the debt of the vault. When the bids pass the debt, the auction reversed, the bidders bid on a lower amount of the deposits on the vault for a specific amount of DAI tokens, specified on the absolute auction step.
  
  \item Liquidation penalty:
The liquidation penalty is an extra punishment for the black coin holders to care about their debt to collateral ratio.
MakerDAO platform charges liquidated vault extra \%13 as a punishment for their vault holders. There are two main reasons to add Liquidation penalty on the design:

\begin{enumerate}

  \item To force vault keepers to be over-collateralized
  \item To mitigate grinding attacks: grinding attack occurs when the position holder deliberately unsafe her black coin and participate in the liquidation auction against her position to buy her deposited assets cheaper.

\end{enumerate}
\end{enumerate}

Using liquidation mechanism as a shield to protect the system provoke criticism. Possibly the black coin holders are freaking out when the price of ETH drops significantly. They may proceed to net out just before the liquidation occurs to their vaults. We called this situation "early liquidation" of the system. 

In this case, the system needs more ETHs to be deposited. The liquidation mechanism is designed to force the black coin holders to inject more ETHs to the system, but the black coin holders withdraw their deposits, and the total collateral of the system drops significantly, which is not the goal of the designers in this situation.

\paragraph{Withholding rewards}

Majority types of designs like liquidation are disincentivizing bad actors. But, another approach is to encourage users to act properly and incentivizing good actors of the system.

For instance, in the Synthetix project users need collateralize SNX tokens (Synthetix network token) to receive sUSD tokens (Synthetix stable coin pegging a USD). There is no margin call methods or liquidation function on the design. But, there is a reward on the system for users who keep more than the over-collateralization ratio. 
The Synthetix system has \%2 annual inflation on SNX tokens. The inflationary tokens are allocated to the vaults that hold more than the collateralization ratio.

There is another reward for the system. The traders on the exchange of the Synthetix project, pay transfer fees collected and distributed to the vaults holding more than the collateral ratio.

The system incentivizes people to stake their SNX token to be over-collateralized and receive the rewards. So, there is no punishment in the system for bad actors in this class of design.

\paragraph{Banning Black coin holders}

In the design of indirectly-backed stable coins, the red coins are not redeemable. In other words, the red coin holders cannot give back their red coins to receive \$1 of the deposited ETH in exchange. The red coin holder must own (or buy if possible) a black coin to net out a vault and receive the ETH.

In the liquidation scenario, the designer pushes black coin holders to be over-collateralized, applying liquidation punishments. Red coin holders and arbitragers are confident that there is no difference between the red coins because each of the red coins is backed by a sufficient amount of ETHs to be \$1 (with high probability).

In another class, the designer removes the liquidation mechanism, prevent black coin holders from withdrawal. In the fungible black coin design class, the designer also forbids the black coin holders from transferring their token. 

In this situation, the incentives for black coin holders to be over-collateralized has been decreased, compared to liquidation design class. But, there is a huge incentive left for them to be over-collateralized. If the price of the underlying asset drops, the black coin holders may want to sell their deposited assets to reduce the loss. In this scenario, just black coin holders that have over-collateralized vaults can net out and receive their underlying assets to sell them to the market.

This type of design will increase the fluctuation of the price of the red coin. The market watches the aggregated collaterals on the system and the number of red coins issued by the system and also the price of the underlying asset to evaluate the price of red coins. So when the price of underlying asset drops, the price of red coins will reduce concerning the underlying asset price. 

In this scenario, red coin holders are taking parts of the risk of underlying asset volatility risk. On the situation that the price of the asset drops significantly, the value of the stable coin will fall.

\paragraph{Reputation systems}

In traditional finance, reputation scoring systems are used to decrease or eliminate the collateral needs for a specific financial transaction. Participants are utilizing their reputation as collateral or source of trust for financial services. 

For instance, in the FICO credit score system, users can enhance their credit limit by increasing their credit score. There is a default risk on credit systems, but the defaulted person will be punished by credit score reduction. The bad actor loses reputation scores forbidden from using plenty of financial services. Therefore, users have adequate incentives to pay their bills.

A revolution of decentralizing the finance products on top of blockchain technologies began in early 2018, named Decentralized Finance (Defi) movement. There is a myriad of different decentralized financial services out there, such as MakerDAO, Compound, Synthetix, Aave, etc. 

There are no differences between users that act properly on Defi platforms and the bad actors. The DeFi ecosystem suffers from a lack of a reputation system or reputation scoring. Using a reputation system will incentivize users to act properly and also reduce the default risk of the system. On the other side of the coin, the users with high-grade reputation scores have new opportunities. So, the cost of defaulting will be increased for high-grade users.

There are barriers to implementing an effective reputation system on blockchains. Lack of strong identities or anonymity is one of them. Also, users can create fake histories. However, these are not impossible to address.

In case that our system concludes a trustworthy reputation system, the designer can use reputation as collateral.  We describe two different designs using reputation systems:
\begin{enumerate}
	\item Reputation-based collateral ratio: 
In the design of the system, the collateral ratio could be reliant on the reputation of the user. In other words, the collateral ratio is higher for new users (users with no reputation) and lower for users that act properly for a long time.
	\item Reputation-based staibility fee
In systems like MakerDAO, the DAI borrowers are obliged to pay a fee on their borrowing named stability fee. This stability fee is being set by Maker token holders.
	In a design based on the reputation, the stability fee could be dependent on the reputation of the user. The reputable user is paying a lower stability fee compared to the new users.
\end{enumerate}
 
\paragraph{Insurance}
Insurance models are used to hedge the risk of unexpected events in different systems. Under-collateralization of a vault is an unexpected event on the RBcoin system. The designer could use an insurance model to protect parties from financial loss in the case of under-collateralization. 

On the insurance model, the insurer will pay a premium to the insurance company. The company will protect the client from financial loss. 

In RBcoins there could be a built-in or outsourced insurance model to protect parties from under collateralization loss. The question raised here is who should pay the premium.

\begin{enumerate}
	\item Premium pay by Red coin holders:
It is very similar to Credit Default Swaps (CDS) on traditional finance. In this design, the approach is that the red coin holders are lending some amount of money to black coin holders. Therefore, black coin holders are borrowing from red coin holders to have a leveraged position on the underlying asset. In this situation, the red coin holders can pay an insurance premium to the contract to protect themselves from the default risk of black coin holders. In the case of under-collateralization, if the black coin holder cannot afford the loss, the insurance contract will pay the loss to the red coin holder.

This type of design is implemented on the MakerDAO platform. The DAI borrowers are paying a premium so-called stability fee to the system. These fees are collected on a pool named Maker Buffer pool. In the case of liquidation of a CDP, if the winner of the auction pays a lower amount of the obligation of the vault, the difference between the obligation and the paid amount will be paid by the Maker Buffer pool.

	\item Premium pay by Black coin holders
In this type of design, the black coin holders are paying the insurance premium. It is similar to regular insurance contracts in which the insurer buys an asset and guarantee it by paying a premium to insurance companies. For example, a person purchases a house and insure it.
Here the black coin holders are buying a position and pay the insurance premium. If the price of underlying asset drops and the vault going to be liquidated the insurance contract will pay on behalf of the insurer.
	\item Premium pay by both
In this scenario, both Red and Black coin holders are paying the insurance premium to ensure their positions. 
\end{enumerate}

\subsubsection{Separate Maturity Dates}
This kind of design employs the idea of Futures in traditional finance. Each contract is an agreement between a volatile and a stable party. They deposit an amount of ETHs on the system (Q). The strike price (K) is the dollar value of ETHs that the parties agree that the stable player will receive at the maturity date (M). The remained ETHs on the vault will go to the pocket of the volatile party.

For a specific amount of pooled ETHs, two tranches are created, the stable token and a volatile token (black token). Tranching in traditional finance is used when several securities are created from a pool of other assets, carrying different risks. The junior tranche (volatile token) takes the majority of the risk and the senior tranche (stable coin) takes a lesser risk.  

The stable tokens are not fungible because each represents a different maturity date.  To create a fungible stable coin, the stable tokens with different maturities are bundled to create the Red coin, stable coin of the system. The amount of red coins each user receives is depending on the maturity date and the strike price of the deposited stable coin.

For example, in the Lien project, the agreement between parties is that at the maturity day, the stable token holder will receive k USD if the deposited ETH worth k USD, and the surplus will belong to the volatile token holder. If the value of deposited ETH dropped below k USD then the stable token worth below K USD and the volatile token worth zero. There are different specified maturity dates every 2 weeks. When a party receives a stable token, she will deposit it on a smart contract named iDOL to receive the stable coin (iDOL token). The iDOL contract bundles stable tokens with different maturities and strike prices and issues a stable coin out of this basket.


\subsection{Black coins Fungibility}

The Black coins, volatile coin of the system, could be fungible or non-fungible. 

\subsubsection{Non-fungible Black coin}

In the majority of implemented indirectly-backed stable coin systems, such as DAI and sUSD black coins are non-fungible. The vaults in these projects are covering various amounts of ETHs. Consequently, the vaults are not fungible. 

Non-fungibility of black coins is one of the primary obstacles of the currently implemented projects. These systems are designed to attract users who need stability along with the features of cryptocurrencies. 

If Alice decides to issue new stable coins, first she requires to create a pair of a red coin and a black coin. She obliged to keep the black coin because black coins are not transferable. So, the users that need stability should wait till another person who is willing to open a leveraged position, create a new vault, and want to sell her red coins to the market.

The other problem of this kind of design is the control of the demand and supply of stable red coins. If the demand for red coins suddenly increases in markets, the price of red coins in markets will be increased. This is an opportunity for arbitragers to make a profit because the price of red coins is pegging a dollar. The arbitragers issue new red coins that cost \$1, sell the red coins to the market to make a profit. But, the problem raised here, because if the arbitrager creates a new vault, then she should hold a non-fungible black coin position. So, the arbitragers are not confident about the price of the red coins. Therefore, there is no motivation for them to make arbitrage on red coin markets.
 
The designers of the MakerDAO platform are using interest rate models to control the supply and demand for red coins and black coins. This core design feature will be explained in detail, in the next sections.

\subsubsection{Fungible Black coin}

Making black coins fungible is the way to solve the mentioned problem. If Alice wants a red coin, she can open a new vault, sell her black coin to the market, and then decide to use it or keep her red coin.

The fungibility of black coins could boost the market capitalization of indirectly-backed stable coins because people who are willing to use stable coins can issue new stable coins without friction.

In this sort of design, there is no need to adjust interest rates to control the demand and supply of red coins because if the price of recoins increases in a market, the arbitragers can create new vaults, sell the issued black coin to the markets and sell the newly generated red coin, which worths a dollar, to a person who is buying them more than a dollar. The arbitrager also could implement these steps in just one transaction, using the meta transaction method to save money on the transaction fees.

The problem with this type of design is when the demand for red coins increases, and there is no demand for black coins. So, the arbitrager should sell the newly generated black coin lower than the issued price.

However, it uses free-market decisions to calculate the price of red and black coins, which means if the price of red coin increases and there is no demand for black coins, the price of red coins will exceed a dollar. 

The other problem with this type of design is the transaction fee for arbitragers. Because the arbitragers are using meta transactions, they should pay high transaction fees for the arbitrage, which may not be profitable in such cases.
\subsection{Underlying asset}
The first decision that a designer should take is choosing the asset that the issuer use as collateral to issue new stable coins.

There is a strong dependency between the risk related to the underlying asset and the design parameters of an indirectly-backed stable coin.

The stable coins could be backed by a single asset, like SAI (the first version of DAI), sUSD, USDx. Or by a basket of different crypto assets like DAI. In multi-collateral backed stable coins, several design parameters depend on the assets used as collateral on the system. 

The purpose of using a basket of assets as collateral is to lower the risk impact of each asset on the stable coin. But the designers may forget the fact that there is a strong correlation between the price of assets. It means as the market of cryptocurrencies falls, all tokens drop in price. 

The underlying assets have two types:

\begin{itemize}
	\item \emph{Exogenous Asset}: Assets that have uses outside of the stable coin systems and just a portion of them are using as collateral on the stable coin system. For instance, ETH, BAT token, and KNC token are collateral options in the MakerDAO platform. These assets are designed to serve other projects, but users can use them as collateral to issue new DAI tokens.
	Another example is Binance token (BNB) used in the USDx protocol. 
	\item \emph{Endogenous Asset}: This class of assets is designed just to be used on the stable coin system. It means the majority of the assets are locked or used on the system. The Synthetix Network Token (SNX) is an example of endogenous assets. The SNX token is created to be used as collateral to issue new sUSD tokens, the stable token in the Synthetix project.
\end{itemize}



\subsection{Supply adjustment}
There is a class of stable coins named Money Supply Adjustment stable coins. In this type of design, the stability comes from the adjustment of the supply of the stable coin. In other words, in the case that the price of stable coin exceeds \$1, the system will increase the amount using a mechanism to reduce the price of the stable coin and vice versa. 

There is a difference between supply adjustment in indirectly-backed mechanism and Money Supply Adjustment method. In Money Supply Adjustment, there is just one coin (Stable coin) that the designer tries to adjust its supply. But, in Indirect-backed stable coins, there are two different tokens, red and black, that their quantity should be modified.

 In a bunch of indirectly-backed stable coins, the designer uses the supply adjustment mechanism to ensure the stable coin's price stability. For instance, in the MakerDAO project, there are two system parameters, Stability fee and Dai Saving Rate (DSR), to adjust the supply of the Dai stable coin.

There is a smart contract named DAI Saving contract or DSR contract. DAI token holders can lock their DAI tokens on the DSR contract and receive interest on the deposited DAI. It is very similar to Saving Accounts in banking. 
The stability fee is the fee that the Dai borrowers must pay back to the system. 

The MakerDAO system uses a combination of these two rates to adjust the DAI tokens and CDPs' supply. The stability fee is the tool to adjust the CDPs (black coins) and the amount of ETHs locked in the system. Decreasing the stability fee is an opportunity for users to create new CDPs with a lower cost of borrowing DAI tokens (red coins). If the Maker token holders conclude that the system needs new vaults, they can reduce the stability fee.
On the other hand, decreasing the DSR rate encourages the users that locked their DAI tokens on the DSR contract to withdraw their tokens and supply them to the market, which increases the supply of the DAI tokens.

The MKR token holders, governance token of the MakerDAO platform, can change. So the supply adjustment on the MakerDAO system is a human intervention mechanism. It has a conflict with the decentralization vision of the platform. Because the stability of the tokens is strongly dependant on the DSR and Stability fee, and humans are setting these two. However, the MakerDAO platform uses Decentralized Autonomous Organizations (DAO) to decentralize governance, but there is a critique, which we will discuss in the governance section.

The question here is, why do we need these two rates to adjust the supply of the DAI tokens. Assume that the demand for DAI tokens increases. The total supply of DAI should be increased, or the DAI price will exceed one dollar due to the supply and demand rule. So, arbitragers have an opportunity to make a profit. They can issue new tokens worth one dollar and sell them on the market, which increases the supply and reduce the price of the DAI token. However, there is a problem with arbitragers. They should create a CDP to issue new DAI tokens, and the CDPs are not transferable. So, the arbitragers cannot issue new tokens in this situation, because they cannot sell the black coins. In this case, the Maker token holders have two choices:
\begin{itemize}
	\item \emph{Reduce the DSR rate}: If a significant amount of DAIs locked in the DSR contract, Maker token holders could vote to reduce the DSR rate. Consequently, the users who locked their tokens on the DSR contract will withdraw their token from the smart contract, the supply of DAI will be increased, and the price will be reduced. 
	\item \emph{ Reduce the Staibility fee}: In case that the deposited amount on the DSR contract is not enough to respond to the demand, new DAI tokens should be issued and supplied to the market to stabilize the DAI price. In such a case, the Maker voters must reduce the stability fee to reduce the cost for users to create new CDPs and DAI tokens. 
	
In this case, the DSR rate is essential, as well. Because it is possible that users create new CDPS and DAIs and then instead of supplying newly generated DAI to the market, deposit them on the DSR contract. So the relation between DSR rate and Stability fee is essential.
\end{itemize}

The critique here is that DAI stable coin is mostly a human intervention-based stable coin. The other solution to supply adjustment is to make Black coins (CDP in MakerDAO protocol) fungible. As discussed before, the arbitragers can issue new tokens if they find arbitrage opportunities and sell the other token (Black coin) to the market. It is a trade-off between the level of intervention on the system and the stability of the Red coins. Because without any involvement, the price of red coins has more fluctuations, but we let the market decide the price. 

The more humans intervene on the system, the more risk of corruption and bribery. The other way to remove humans' involvement in the system is to replace it with automated mechanisms. The designers can use the idea behind Money Supply Adjustment models to automate this part or invent a method to change the system's rates automatically.

\subsection{Governance}
The decisions about the future of the project are of importance in the design of the system. There is a spectrum of governance models for the future of the project. The right spot of the spectrum is when the project's founders make the proposals and changes in the system. This method is the most centralized type of governance design. The other side of the spectrum is when the system does not use governance models, i.e., the developers deploy the code to the blockchain and leave the project. So, there won't be changed in the future.

In the middle of the spectrum,  the designer tries to decentralized the governance of the project. In these projects, the creators distribute governance tokens by a mechanism such as Initial Coin Offering (ICO) or Yield Farming. Then, the governance token holders are responsible for voting on the change proposals of the systems.

For instance, the MKR token is the governance token of the Maker system. Each token represents a vote for future proposals. There are critiques about the governance model of the Maker platform:

\begin{itemize}
	\item \emph{Technocracy instead of Democracy}: There is a debate on this type of design about information asymmetry. The governance token holders who gain from the technical background has more information about the smart contracts, processes, and logic behind the protocol. This information asymmetry could help the professional voters get their way on the proposals and votes. In other words, for plans that have benefits for them, they use their knowledge to convince the other voters to vote on the proposals.
	
The ordinary users in these systems will follow the technocrats on the voting, and it gives the technocrats higher decision power than what they have on their pockets.   
	\item \emph{Level of Centralization}: One of the key points in designing a DAO for the governance of a project is the voters' level of decentralism. The privacy characteristic of the blockchains makes it hard to track the identity of token owners. If a malicious user owns a significant portion of the governance token, then the system is susceptible to governance attacks, and the malicious user can vote and execute the proposals to maximize the profit. 
\end{itemize}

\subsection{Implications on Regulatory}


% = = = = = Bibliography = = = = = %

\bibliography{bib/pulp.bib,bib/new.bib}
%\nocite{*}

% = = = = = End Notes = = = = = %

\clearpage
\appendix

%\section{Definitions}
%
%\paragraph{Leverage.} Alice and Bob each have \$100. Alice buys 1 share of APPL, while Bob who buys 1 share of APPL and takes a \$900 loan to buy another 9 shares (assume the loan is free). APPL increases from \$100 to \$200. Alice is now worth \$200 and Bob is worth \$2000. Next, assume APPL decreases in value to \$90. Alice is worth \$90 but Bob is worth \$0 (his shares are worth \$900 which is the same as the loan). If APPL goes to \$85, Bob is insolvent. Alice and Bob both profit when APPL increases in value and both lose money when it decreases, but Bob's gains/losses are amplified by a 10:1 ratio. This is called leverage. Many exchanges (in both traditional and crypto-markets) offer credit for leveraged trading. It is also known as margin trading---the exchange does not want Bob to become insolvent and owe it money that he might never pay, so the exchange forces Bob to deposit a certain amount of money (called margin) with the exchange to cover any loses. If margin account approaches zero and Bob is unable, unwilling, or does not have enough time to top it up, the exchange will close all of his financial positions and repay the loan. 

% = = = = = = = = = = = = = = = = = = = = = = = = = = = = = = = = = = = = = = = = = =

\end{document}







