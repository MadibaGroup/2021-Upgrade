% !TEX root = ../main.tex
\section{Upgrading process}
 
% \subsection{Level of Decentralization}

% The last and one of the most important characteristics that are different in upgradeability methods is the level of decentralization. An upgradeability methods that a single third party decides the upgrading process receives a square (\XBox). 
% Using an \textbf{EOA} to decide about a change is the most central option that a system designer can choose regardless of the upgradeability method uses in the system. 
% In case a group of whitelisted persons can decide on the changes orf the system using \textbf{Multi-sig} is not decentralized as well. Although it improves the level of decentralization of the system but at the end a specified number can decide to change the system. So it awarded an empty circle (\Circle).

% Utilizing a decentralized governance model to vote for a change is a good way to make the decision making on the upgrades more decentralized. \textit{Retail Changes} using voting scheme is more decentral than \textit{Call-based} and \textit{DelegateCall-based} because boundary of changes are limited on the Retail methods so it awarded a full circle (\CIRCLE). But, in CAll-based and DelegateCall based methods the developers have the power to put some kind of backdoor in the system while upgrading and they receive a half circle (\LEFTcircle). 

% The \textit{Migration} method is the most decentralized approach because it gives the users chance to decide whether to move to the newer version or not so it awarded a full circle (\CIRCLE). For instance, Uniswap uses this method for its upgrade and the users have choice to transfer their funds from Uniswap V2 to V3 or not and as we can see some users decide to stay on the previous version.


\subsection{Mitigating risks}
There are critical setups on the systems to mitigate the possible risks on the upgrading process. We mention some of them here with risk associated with them.

\subsubsection{Timelocks}
In some project, there is a time window between every changes that approved on the system and when they affect the system. This gives opportunity to the users who are not satisfied with the upcoming upgrades to move their funds out of the system. However this is not proper in case of fixing a bug, because we need to patch the problem quickly.

\subsubsection{Thereshold}
In multi-sig and governance upgrade methods we need a threshold on votes to decide whether a change is approved or not. This threshold should be big enough to be confident that upgrading event represents the majority of opinions. On the other hand, the threshold shouldn't be that big because a big threshold will delay a system change. The system designer should consider that a portion of voters (signers in multi-sig or governance token holders in voting method) may not be available in the event of the upgrade and having a big threshold may result in halting the change proposal for a long period of time. In fact threshold has a trade-off between security/decentralization and speed of the upgrade process.

\subsubsection{Pausable}
In pauseable smart contracts, the decision makers (usually a multi-sig wallet) can freeze some or all operations of the system. Pausing a smart contract helps in some specific situations:

\begin{enumerate}
  \item Time to react to a bug or hack: usually it takes time to analyze and find the reason of a hack and patch the bug. In this time period the core developer team needs to pause the system to stop attacker from draining all the fund.
  \item Halting system in the upgrade process: For instance, in an ERC20 token contract upgrade we need to pause the system to stop users from transferring tokens during the upgrade. 
  \item Inactivating the previous version of the logic contracts: After an upgrade we need to have a plan to stop users from using the previous logic contracts. One way to do so is to make the logic contracts pauseable and pause them after the upgrade. 
\end{enumerate}

\subsubsection{Escape Hatches}
A escape hatch is a mechanism that lets the users to move their fund out of the system in the pausing events. For instance, in MakerDAO we have an emergency shutdown mechanism that pauses the system in the black swan events. But, users have the ability to extract their funds out of the system while the system is paused.
\subsubsection{Front-Runnign}
Upgrading a smart contract can be done by sending a transaction into the system. If the upgrade is a response to a unknown bug, then the upgrade process will hint attackers who is listening to the mempool to find the bug and hack the smart contract just before the upgrade. So there should be some mechanisms to mitigate front-running attacks. One solution to this issue is to use commit-reveal schemes. The team first sends a commitment of the upgrade (hash of the upgrade) to the system and after the timelock they can push and apply the original code which cannot be front run. 
%Commit reveal



% upgrading in dark: https://forum.makerdao.com/t/mip15-dark-spell-mechanism/2578
